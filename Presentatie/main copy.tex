%----------------------------------------------------------------------------------------
%	PACKAGES AND THEMES
%----------------------------------------------------------------------------------------
\documentclass[aspectratio=169]{beamer}
%\usetheme{Simple}



\usepackage{lmodern}
 \setbeamercovered{transparent}
 \usetheme[width=4\baselineskip,hideothersubsections]{Berkeley}

 \setbeamerfont{section in sidebar}{size=\scriptsize}
 \setbeamerfont{subsection in sidebar}{size=\tiny}

 \usepackage{tikz}
 \usetikzlibrary{external}
 

 \usepackage{braket}

% \usepackage{graphicx} % Allows including images

% \usepackage{subcaption} % Subfigure environment 
% \usepackage{gensymb}

 \usepackage{amsmath} % Mathematical symbols
\usepackage{amssymb} % Symbols

% \usepackage{caption}% Captions onder figuur gecentreerd
% %\usepackage[toc,page]{appendix}
% \usepackage{subcaption} % Subfigure environment 
% \usepackage{float}

 \usepackage{etoolbox} %for if empty functionality
 \usepackage{ifthen}

% %break long urls at a - and not only at . or /
 \usepackage{hyperref}

% \usepackage{verbatim}
% \usepackage{siunitx} % Elegant eenheden zetten
% \usepackage[version=3]{mhchem} % ingeven van chemische fomules
\usepackage{cleveref} % Paragraaf tekens
% \usepackage{longtable}
% %\usepackage{lscape}
% \usepackage[T1]{fontenc}
 \usepackage{amsfonts}
 \usepackage{mathtools}
% \usepackage{grffile}%double dot in figure name
% \usepackage{lipsum}
% \usepackage{siunitx}
% \usepackage{xcolor}
% \usepackage{sectsty}
 \usepackage{booktabs}
 \usepackage{physics}
% \usepackage{leftidx}
 \usepackage[utf8]{inputenc}
% \usepackage{gensymb} % /textdegree
%\def\temp{#1}\ifx\temp\empty
%  <EMPTY>%
%\else
%  <NON EMPTY>%
%\fi




%use first argument is numbers of O's, second the labels, can be empty
%\mpo{3}{ {0,1,2,3,4} }

\newcommand{\combineTikz}[3]{
    \begin{tikzpicture}[baseline={0-0.5*height("$=$")}]
        \node (AA) at (0,0)  { #1   };
        \node (AB) at ( {#3} ,0)  {  #2  };
    \end{tikzpicture}
}


\newcommand{\mpo}[6]  { \begin{tikzpicture}[baseline={0-0.5*height("$=$")}]
        %\def \NNodes {#1}
        %\def \NodeName {#2}          
        %\def \NameUp   {#3} 
        %\def \NameDown  {#4}	

        \def \legLength {0.6}
        \def \radius {0.3}

        \pgfmathsetmacro{\step}{2*\radius+\legLength}
        \pgfmathsetmacro{\legpos}{\radius+\legLength}

        \pgfmathsetmacro{\Nmax}{#1-1}

        \foreach \N in {0,..., \Nmax }{
                \pgfmathsetmacro{\p}{\N*\step}

                % up and down labels
                \def\temp{#3}\ifx\temp\empty
                    \def \labelUp {}
                \else
                    \pgfmathsetmacro{\labelUp}{  {#3}[\N]  }
                \fi

                \def\tempp{#4}\ifx\tempp\empty
                    \def \labeldown {}
                \else
                    \pgfmathsetmacro{\labeldown}{  {#4}[\N]  }
                \fi


                \def\aab{#5}\ifx\aab\empty
                    \def \dotssite {0}
                \else
                    \pgfmathsetmacro{\dotssite}{  {#5}[\N]  }
                \fi

                \ifthenelse{\dotssite = 0}{

                    \def\aac{#6}\ifx\aac\empty
                        \def \nname {O}
                    \else
                        \pgfmathsetmacro{\nname}{  {#6}[\N]  }
                    \fi



                    \node[circle,draw, radius=\radius] (O\N) at (\p,0) {\nname};


                    \node[] (Ou\N) at (\p, \legpos ) { \labelUp };


                    \node[] (Od\N) at (\p,-\legpos) {\labeldown};


                    \draw (O\N) -- (Ou\N);
                    \draw (O\N) -- (Od\N);
                }{
                    \node[circle] (O\N) at (\p,0) { $\cdots$ };


                }


            }

        \ifthenelse{  #1  =1  }{}{
            \foreach \N in {1,...,\Nmax }{
                    \pgfmathsetmacro{\M}{\N-1}
                    \pgfmathsetmacro{\label}{ {#2}[\N]  }
                    %\pgfmathsetmacro{\label}{ 5}

                    \draw (O\M) --  node[above]  {\label} (O\N);
                }
        }

        \pgfmathsetmacro{\labelo}{ {#2}[0]}
        \pgfmathsetmacro{\labeli}{  {#2}[\Nmax+1]}

        \node (N0) at (-\legpos,0) {};
        \draw (N0) -- node[above] {\labelo} (O0);

        \pgfmathsetmacro{\endpos}{\step*\Nmax+\legpos}

        \node (Ne) at (\endpos,0) {};
        \draw (Ne) -- node[above] {\labeli} (O\Nmax);

        %\draw (O0) --  node[above] {1} (O1);


    \end{tikzpicture}}

\newcommand{\expH}[5]{ \begin{tikzpicture}[baseline={0-0.5*height("$=$")}]

        \def \NNodes {#1};


        \def\aaa{#2}\ifx\aaa\empty
            \def \text { $e^{-\beta \hat{H}_{\NNodes} }$ }
        \else
            \def \text {#2}
        \fi

        \pgfmathwidth{ "\text" }
        \def \textwidth { \pgfmathresult }


        %\pgfmathsetmacro{\text}{width(\text)}

        \def \legLength {0.6}
        \def \radius {0.3} %fix to fit text inside for size 1
        \def \boxHeight {0.4};

        \pgfmathsetmacro{\step}{2*\radius+\legLength}
        \pgfmathsetmacro{\legpos}{\radius+\legLength}
        \pgfmathsetmacro{\dotpos}{\boxHeight+\legLength/2}

        \pgfmathsetmacro{\Nmax}{\NNodes -1}



        \pgfmathsetmacro{\boxsize}{ max ( \textwidth/1cm , \step*\Nmax )   + \radius}

        %\pgfmathsetmacro{\boxsize}{ 5  )}
        %\pgfmathsetlength{\boxsize}{ max( \textwidth,  \boxsize1  )}



        %            \ifthenelse{#1=1}{
        %                \def \left {-0.6}
        %                \def \right {0.6}
        %            }{
        \def \left {-\radius}
        \def \right {\boxsize}
        %            }

        \draw (\left,- \boxHeight ) rectangle (\right, \boxHeight ) [add reference =H] ;


        \node  at (H center) { \text };

        \foreach \N in {0,..., \Nmax }{
                \pgfmathsetmacro{\p}{\N*\step}

                % up and down labels
                \def\temp{#3}\ifx\temp\empty
                    \def \labelUp {}
                \else
                    \pgfmathsetmacro{\labelUp}{  {#3}[\N]  }
                \fi

                \def\tempp{#4}\ifx\tempp\empty
                    \def \labeldown {}
                \else
                    \pgfmathsetmacro{\labeldown}{  {#4}[\N]  }
                \fi


                \node[] (O\N) at (\p,0) {};


                \ifthenelse{ \equal{\labelUp}{...}  }{
                    \node[] (Ou\N) at (\p, \dotpos ) {\labelUp};
                }{
                    \node[] (Ou\N) at (\p, \legpos ) {\labelUp};
                    \draw (Ou\N) --  (Ou\N  |- H north);
                }


                \ifthenelse{ \equal{\labeldown}{...}  }{
                    \node[] (Od\N) at (\p,-\dotpos ) {\labeldown};
                }{
                    \node[] (Od\N) at (\p,-\legpos) {\labeldown};
                    \draw (Od\N) --  (Od\N  |- H south);
                }
            }

        \def\tempt{#5}\ifx\tempt\empty

        \else
            \pgfmathsetmacro{\labelo}{ {#5}[0] }
            \pgfmathsetmacro{\labeli}{  {#5}[1] }

            \pgfmathsetmacro{\leftleg}{  \left - \legLength }
            \pgfmathsetmacro{\rightleg}{  \right + \legLength }

            \node (N0) at (\leftleg,0) {\labelo};
            \draw (N0) -- ( N0  -| H west);

            \node (Ne) at (\rightleg,0) {\labeli};
            \draw (Ne) --  ( Ne  -| H east);
        \fi


    \end{tikzpicture} }



%----------------------------------------------------------------------------------------
%	TITLE PAGE
%----------------------------------------------------------------------------------------

% The title
\title[]{PEPO cluster expansion of Tensor Exponential  }
\subtitle{Subtitle}

\author[] {David Devoogdt}
\institute[UGent] % Your institution may be shorthand to save space
{
    % Your institution for the title page
    Faculty of Engineering and Architecture \\
    Ghent University 
    \vskip 3pt
}
\date{\today} % Date, can be changed to a custom date


%----------------------------------------------------------------------------------------
%	PRESENTATION SLIDES
%----------------------------------------------------------------------------------------



\begin{document}

\begin{frame}
    % Print the title page as the first slide
    \titlepage
\end{frame}

\begin{frame}
    % Print the title page as the first slide
    \tableofcontents[hidesubsections]
\end{frame}

\AtBeginSection[]
{
    \begin{frame}<beamer>{Table of Contents}
        \tableofcontents[currentsection,currentsubsection,
            hideothersubsections,
            sectionstyle=show/shaded,
        ]
    \end{frame}
}

%------------------------------------------------
%------------------------------------------------
\section{Intoduction}
%------------------------------------------------
%------------------------------------------------


%------------------------------------------------
\subsection{Problem Statement}
%------------------------------------------------
\begin{frame}{Statistical Quantum mechanics}
    \begin{equation}
        \hat{\rho} =  \frac{ e^{ - \beta \hat{H} } }{Z}
    \end{equation}

    \begin{equation}
        \begin{split}
            Z &= \Tr(  e^{ - \beta \hat{H} ) \\
            \Braket{X} &= \Tr(\rho \hat{X})
        \end{split}
    \end{equation}
\end{frame}



%------------------------------------------------
\subsection{Graphical notation}
%------------------------------------------------

\begin{frame}{Graphical notation}

    \begin{equation}
        \mpo{1}{ {0,0}  }{ {"$i$",}  }{ {"$j$",}}{}{{"",}} = \mpob{1}{ {0,0}  }{ {"$i$",}  }{ {"$j$",}}{}{{"",}}
    \end{equation}

    \begin{equation}
        \mpo{2}{ {0,1,0}  }{ {"$i_1$","$i_2$"}  }{ {"$j_1$","$j_2$",}}{}{{"",}} =\mpob{2}{ {0,1,0}  }{ {"$i_1$","$i_2$"}  }{ {"$j_1$","$j_1$",}}{}{{"",}}
    \end{equation}

    \begin{equation}
        \mpo{3}{ {0,,,0}  }{ {"$i_1$","$i_2$","$i_3$"}  }{ {"$j_1$","$j_2$","$j_3$"}}{}{{"",,}} =\mpob{3}{ {0,,,0}  }{}{}{}{{"",,}}
    \end{equation}

\end{frame}

\begin{frame}{Graphical notation}
    \begin{equation}
        \hat{H} = \left (  \sum_{<i j>} H^i_2 H^j_2 + \sum_i H^i_1 \right )
    \end{equation}

    \begin{equation}
        \begin{split}
            H \left( \mpob{3}{ {0,,,0}  }{}{}{}{{"",,}} \right ) = &H_1 \otimes 1 \otimes 1 \\
            +  &1 \otimes H_1  \otimes 1 \\
            +  &1 \otimes 1 \otimes H_1   \\
            +  &H_2 \otimes H_2 \otimes 1  \\
            +  &1 \otimes H_2 \otimes H_2  \\
        \end{split}
    \end{equation}
\end{frame}



%------------------------------------------------
\subsection{Cluster expansion}
%------------------------------------------------
\begin{frame}{General idea}
    \begin{itemize}
        \item represent as MPO/PEPO
        \item cluster by size, not in $\beta$
    \end{itemize}
\end{frame}







\begin{frame}{General idea}
    \begin{equation}
        \mpob{1}{ {0,0}  }{}{}{}{{"",}} = \exp \left( -\beta H(\mpob{1}{}{}{}{}{{"",}})   \right)
    \end{equation}

    \begin{equation}
        \begin{split}
            \mpob{2}{ {0,1,0}  }{}{}{}{{"",}}  = \exp -\beta H( & \mpob{2}{ {,,} }{}{}{}{{"",}}) \\
            -&\mpob{2}{ {0,0,0}  }{}{}{}{{"",}}
        \end{split}
    \end{equation}

\end{frame}

\begin{frame}{General idea}

    \begin{equation}
        \begin{split}
            \mpob{3}{ {0,1,1,0}  }{}{}{}{{,,,}}  = \exp -\beta H( &\mpob{3}{ {,,,} }{}{}{}{{,,}})  \\
            -&\mpob{3}{ {0,0,0,0}  }{}{}{}{{,,,}}\\
            -&\mpob{3}{ {0,1,0,0}  }{}{}{}{{,,,}}\\
            -&\mpob{3}{ {0,0,1,0}  }{}{}{}{{,,,}}
        \end{split}
    \end{equation}

\end{frame}


%=  \exp \left(  -\beta \mpob{1}{{,,}}{}{}{}{{"H","H",}} \right)
%=  \exp \left(  -\beta \mpob{2}{{,,}}{}{}{}{{"H","H",}} \right)



\begin{frame}{Advantages}
    \begin{itemize}
        \item size extensive
        \item symmetry
    \end{itemize}
\end{frame}

%------------------------------------------------
%------------------------------------------------
\section{Solvers}
%------------------------------------------------
%------------------------------------------------

%------------------------------------------------
\subsection{Problem statement}
%------------------------------------------------
\begin{frame}{Bullet Points}

\end{frame}

%------------------------------------------------
\subsection{Structure framework}
%------------------------------------------------
\begin{frame}{map}

\end{frame}

\begin{frame}{map}

\end{frame}

%------------------------------------------------
\subsection{Numerical considerations}
%------------------------------------------------
\begin{frame}{Normalisation}

\end{frame}

\begin{frame}{Single tensor or cells}

\end{frame}

\begin{frame}{Fast cell contraction}

\end{frame}

%------------------------------------------------
\subsection{Linear solvers}
%------------------------------------------------
\begin{frame}{Bullet Points}

\end{frame}

\begin{frame}{Simplifying inverse?}

\end{frame}

%------------------------------------------------
\subsection{Non-linear solvers}
%------------------------------------------------
\begin{frame}{Bullet Points}

\end{frame}
%------------------------------------------------
\subsection{sequential linear solvers}
%------------------------------------------------
\begin{frame}{Bullet Points}

\end{frame}



%------------------------------------------------
%------------------------------------------------
\section{Construction 1D}
%------------------------------------------------
%------------------------------------------------
\begin{frame}{Bullet Points}

\end{frame}


%------------------------------------------------
\subsection{Variant 1}
%------------------------------------------------
\begin{frame}{Bullet Points}

\end{frame}
%------------------------------------------------
\subsection{Variant 2}
%------------------------------------------------
\begin{frame}{Bullet Points}

\end{frame}
%------------------------------------------------
\subsection{Results}
%------------------------------------------------
\begin{frame}{Bullet Points}

\end{frame}
%------------------------------------------------
%------------------------------------------------
\section{Construction 2D}
%------------------------------------------------
%------------------------------------------------
\begin{frame}
    \begin{equation}
        \exp H \left(  \pepob{2}{2}{{,,}}{{,,}}{{0,0,0,1}} \right)
    \end{equation}
\end{frame}

\begin{frame}
    \begin{equation}
        \pepob{2}{3}{{"1","2","3"}}{{"4","5","6","7",} }{}
    \end{equation}
\end{frame}


%------------------------------------------------
\subsection{Generalising 1D construction}
%------------------------------------------------
\begin{frame}{Bullet Points}

\end{frame}
%------------------------------------------------
\subsection{Loops}
%------------------------------------------------
\begin{frame}{Bullet Points}

\end{frame}
%------------------------------------------------
%------------------------------------------------
\section{2D Ising Model}
%------------------------------------------------
%------------------------------------------------

%------------------------------------------------
\subsection{Overview}
%------------------------------------------------
\begin{frame}{Bullet Points}

\end{frame}
%------------------------------------------------
\subsection{First results}
%------------------------------------------------
\begin{frame}{Bullet Points}

\end{frame}

%------------------------------------------------
%------------------------------------------------
\section{Outlook}
%------------------------------------------------
%------------------------------------------------
\begin{frame}{Bullet Points}

\end{frame}

\end{document}
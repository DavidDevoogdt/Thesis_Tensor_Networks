%----------------------------------------------------------------------------------------
%	PACKAGES AND THEMES
%----------------------------------------------------------------------------------------
\documentclass[aspectratio=169]{beamer}
%\usetheme{Simple}

\usepackage{listings}
\usepackage{lmodern}
 \setbeamercovered{transparent}
 \usetheme[width=4\baselineskip,hideothersubsections]{Berkeley}

 \setbeamerfont{section in sidebar}{size=\scriptsize}
 \setbeamerfont{subsection in sidebar}{size=\tiny}

 \usepackage{tikz}
 \usetikzlibrary{external}
 

 \usepackage{braket}

% \usepackage{graphicx} % Allows including images

% \usepackage{subcaption} % Subfigure environment 
% \usepackage{gensymb}

 \usepackage{amsmath} % Mathematical symbols
\usepackage{amssymb} % Symbols

% \usepackage{caption}% Captions onder figuur gecentreerd
% %\usepackage[toc,page]{appendix}
% \usepackage{subcaption} % Subfigure environment 
% \usepackage{float}

 \usepackage{etoolbox} %for if empty functionality
 \usepackage{ifthen}

% %break long urls at a - and not only at . or /
 \usepackage{hyperref}

% \usepackage{verbatim}
% \usepackage{siunitx} % Elegant eenheden zetten
% \usepackage[version=3]{mhchem} % ingeven van chemische fomules
\usepackage{cleveref} % Paragraaf tekens
% \usepackage{longtable}
% %\usepackage{lscape}
% \usepackage[T1]{fontenc}
 \usepackage{amsfonts}
 \usepackage{mathtools}
% \usepackage{grffile}%double dot in figure name
% \usepackage{lipsum}
% \usepackage{siunitx}
% \usepackage{xcolor}
% \usepackage{sectsty}
 \usepackage{booktabs}
 \usepackage{physics}
% \usepackage{leftidx}
 \usepackage[utf8]{inputenc}
% \usepackage{gensymb} % /textdegree


%----------------------------------------------------------------------------------------
%	TITLE PAGE
%----------------------------------------------------------------------------------------

% The title
\title[]{PEPO cluster expansion of tensor exponential  }

\author[] {David Devoogdt}
\institute[UGent] % Your institution may be shorthand to save space
{
    % Your institution for the title page
    Faculty of Engineering and Architecture \\
    Ghent University 
    \vskip 3pt
}
\date{\today} % Date, can be changed to a custom date


%\def\temp{#1}\ifx\temp\empty
%  <EMPTY>%
%\else
%  <NON EMPTY>%
%\fi




%use first argument is numbers of O's, second the labels, can be empty
%\mpo{3}{ {0,1,2,3,4} }

\newcommand{\combineTikz}[3]{
    \begin{tikzpicture}[baseline={0-0.5*height("$=$")}]
        \node (AA) at (0,0)  { #1   };
        \node (AB) at ( {#3} ,0)  {  #2  };
    \end{tikzpicture}
}


\newcommand{\mpo}[6]  { \begin{tikzpicture}[baseline={0-0.5*height("$=$")}]
        %\def \NNodes {#1}
        %\def \NodeName {#2}          
        %\def \NameUp   {#3} 
        %\def \NameDown  {#4}	

        \def \legLength {0.6}
        \def \radius {0.3}

        \pgfmathsetmacro{\step}{2*\radius+\legLength}
        \pgfmathsetmacro{\legpos}{\radius+\legLength}

        \pgfmathsetmacro{\Nmax}{#1-1}

        \foreach \N in {0,..., \Nmax }{
                \pgfmathsetmacro{\p}{\N*\step}

                % up and down labels
                \def\temp{#3}\ifx\temp\empty
                    \def \labelUp {}
                \else
                    \pgfmathsetmacro{\labelUp}{  {#3}[\N]  }
                \fi

                \def\tempp{#4}\ifx\tempp\empty
                    \def \labeldown {}
                \else
                    \pgfmathsetmacro{\labeldown}{  {#4}[\N]  }
                \fi


                \def\aab{#5}\ifx\aab\empty
                    \def \dotssite {0}
                \else
                    \pgfmathsetmacro{\dotssite}{  {#5}[\N]  }
                \fi

                \ifthenelse{\dotssite = 0}{

                    \def\aac{#6}\ifx\aac\empty
                        \def \nname {O}
                    \else
                        \pgfmathsetmacro{\nname}{  {#6}[\N]  }
                    \fi



                    \node[circle,draw, radius=\radius] (O\N) at (\p,0) {\nname};


                    \node[] (Ou\N) at (\p, \legpos ) { \labelUp };


                    \node[] (Od\N) at (\p,-\legpos) {\labeldown};


                    \draw (O\N) -- (Ou\N);
                    \draw (O\N) -- (Od\N);
                }{
                    \node[circle] (O\N) at (\p,0) { $\cdots$ };


                }


            }

        \ifthenelse{  #1  =1  }{}{
            \foreach \N in {1,...,\Nmax }{
                    \pgfmathsetmacro{\M}{\N-1}
                    \pgfmathsetmacro{\label}{ {#2}[\N]  }
                    %\pgfmathsetmacro{\label}{ 5}

                    \draw (O\M) --  node[above]  {\label} (O\N);
                }
        }

        \pgfmathsetmacro{\labelo}{ {#2}[0]}
        \pgfmathsetmacro{\labeli}{  {#2}[\Nmax+1]}

        \node (N0) at (-\legpos,0) {};
        \draw (N0) -- node[above] {\labelo} (O0);

        \pgfmathsetmacro{\endpos}{\step*\Nmax+\legpos}

        \node (Ne) at (\endpos,0) {};
        \draw (Ne) -- node[above] {\labeli} (O\Nmax);

        %\draw (O0) --  node[above] {1} (O1);


    \end{tikzpicture}}

\newcommand{\expH}[5]{ \begin{tikzpicture}[baseline={0-0.5*height("$=$")}]

        \def \NNodes {#1};


        \def\aaa{#2}\ifx\aaa\empty
            \def \text { $e^{-\beta \hat{H}_{\NNodes} }$ }
        \else
            \def \text {#2}
        \fi

        \pgfmathwidth{ "\text" }
        \def \textwidth { \pgfmathresult }


        %\pgfmathsetmacro{\text}{width(\text)}

        \def \legLength {0.6}
        \def \radius {0.3} %fix to fit text inside for size 1
        \def \boxHeight {0.4};

        \pgfmathsetmacro{\step}{2*\radius+\legLength}
        \pgfmathsetmacro{\legpos}{\radius+\legLength}
        \pgfmathsetmacro{\dotpos}{\boxHeight+\legLength/2}

        \pgfmathsetmacro{\Nmax}{\NNodes -1}



        \pgfmathsetmacro{\boxsize}{ max ( \textwidth/1cm , \step*\Nmax )   + \radius}

        %\pgfmathsetmacro{\boxsize}{ 5  )}
        %\pgfmathsetlength{\boxsize}{ max( \textwidth,  \boxsize1  )}



        %            \ifthenelse{#1=1}{
        %                \def \left {-0.6}
        %                \def \right {0.6}
        %            }{
        \def \left {-\radius}
        \def \right {\boxsize}
        %            }

        \draw (\left,- \boxHeight ) rectangle (\right, \boxHeight ) [add reference =H] ;


        \node  at (H center) { \text };

        \foreach \N in {0,..., \Nmax }{
                \pgfmathsetmacro{\p}{\N*\step}

                % up and down labels
                \def\temp{#3}\ifx\temp\empty
                    \def \labelUp {}
                \else
                    \pgfmathsetmacro{\labelUp}{  {#3}[\N]  }
                \fi

                \def\tempp{#4}\ifx\tempp\empty
                    \def \labeldown {}
                \else
                    \pgfmathsetmacro{\labeldown}{  {#4}[\N]  }
                \fi


                \node[] (O\N) at (\p,0) {};


                \ifthenelse{ \equal{\labelUp}{...}  }{
                    \node[] (Ou\N) at (\p, \dotpos ) {\labelUp};
                }{
                    \node[] (Ou\N) at (\p, \legpos ) {\labelUp};
                    \draw (Ou\N) --  (Ou\N  |- H north);
                }


                \ifthenelse{ \equal{\labeldown}{...}  }{
                    \node[] (Od\N) at (\p,-\dotpos ) {\labeldown};
                }{
                    \node[] (Od\N) at (\p,-\legpos) {\labeldown};
                    \draw (Od\N) --  (Od\N  |- H south);
                }
            }

        \def\tempt{#5}\ifx\tempt\empty

        \else
            \pgfmathsetmacro{\labelo}{ {#5}[0] }
            \pgfmathsetmacro{\labeli}{  {#5}[1] }

            \pgfmathsetmacro{\leftleg}{  \left - \legLength }
            \pgfmathsetmacro{\rightleg}{  \right + \legLength }

            \node (N0) at (\leftleg,0) {\labelo};
            \draw (N0) -- ( N0  -| H west);

            \node (Ne) at (\rightleg,0) {\labeli};
            \draw (Ne) --  ( Ne  -| H east);
        \fi


    \end{tikzpicture} }


\tikzset{add reference/.style={insert path={%
                    coordinate [pos=0,xshift=-0.5\pgflinewidth,yshift=-0.5\pgflinewidth] (#1 south west)
                    coordinate [pos=1,xshift=0.5\pgflinewidth,yshift=0.5\pgflinewidth]   (#1 north east)
                    coordinate [pos=.5] (#1 center)
                    (#1 south west |- #1 north east)     coordinate (#1 north west)
                    (#1 center     |- #1 north east)     coordinate (#1 north)
                    (#1 center     |- #1 south west)     coordinate (#1 south)
                    (#1 south west -| #1 north east)     coordinate (#1 south east)
                    (#1 center     -| #1 south west)     coordinate (#1 west)
                    (#1 center     -| #1 north east)     coordinate (#1 east)
                }}}

\setbeamertemplate{footline}[frame number]

%----------------------------------------------------------------------------------------
%	PRESENTATION SLIDES
%----------------------------------------------------------------------------------------



\begin{document}

\begin{frame}
    % Print the title page as the first slide
    \titlepage
\end{frame}

% \begin{frame}
%     % Print the title page as the first slide
%     \tableofcontents[hidesubsections]
% \end{frame}

\AtBeginSection[]
{
    \begin{frame}<beamer>{Table of Contents}
        \tableofcontents[currentsection,currentsubsection,
            hideothersubsections,
            sectionstyle=show/shaded,
        ]
    \end{frame}
}

\AtBeginSection[]{
    \begin{frame}
        \vfill
        \centering
        \begin{beamercolorbox}[sep=8pt,center,shadow=true,rounded=true]{title}
            \usebeamerfont{title}\insertsectionhead\par%
        \end{beamercolorbox}
        \vfill
    \end{frame}
}

%------------------------------------------------
%------------------------------------------------
\section{Intoduction}
%------------------------------------------------
%------------------------------------------------


%------------------------------------------------
\subsection{Problem Statement}
%------------------------------------------------
\begin{frame}{Statistical Quantum mechanics}
    \begin{equation}
        \hat{\rho} =  \frac{ e^{ - \beta \hat{H} } }{Z}
    \end{equation}

    \begin{equation}
        \begin{split}
            Z &= \Tr(  e^{ - \beta \hat{H} } ) \\
            \Braket{X} &= \Tr(\rho \hat{X})
        \end{split}
    \end{equation}
\end{frame}



%------------------------------------------------
\subsection{Graphical notation}
%------------------------------------------------

\begin{frame}{Graphical notation}

    \begin{equation}
        \mpo{1}{ {0,0}  }{ {"$i$",}  }{ {"$j$",}}{}{{"",}} = \mpob{1}{ {0,0}  }{ {"$i$",}  }{ {"$j$",}}{}{{"",}}
    \end{equation}

    \begin{equation}
        \mpo{2}{ {0,1,0}  }{ {"$i_1$","$i_2$"}  }{ {"$j_1$","$j_2$",}}{}{{"",}} =\mpob{2}{ {0,1,0}  }{ {"$i_1$","$i_2$"}  }{ {"$j_1$","$j_1$",}}{}{{"",}}
    \end{equation}

    \begin{equation}
        \mpo{3}{ {0,,,0}  }{ {"$i_1$","$i_2$","$i_3$"}  }{ {"$j_1$","$j_2$","$j_3$"}}{}{{"",,}} =\mpob{3}{ {0,,,0}  }{}{}{}{{"",,}}
    \end{equation}

\end{frame}

\begin{frame}{Graphical notation}
    \begin{equation}
        \hat{H} = \left (  \sum_{<i j>} H^i_2 H^j_2 + \sum_i H^i_1 \right )
    \end{equation}

    \begin{equation}
        \begin{split}
            H \left( \mpob{3}{ {0,,,0}  }{}{}{}{{"",,}} \right ) = &H_1 \otimes 1 \otimes 1 \\
            +  &1 \otimes H_1  \otimes 1 \\
            +  &1 \otimes 1 \otimes H_1   \\
            +  &H_2 \otimes H_2 \otimes 1  \\
            +  &1 \otimes H_2 \otimes H_2  \\
        \end{split}
    \end{equation}
\end{frame}





%------------------------------------------------
\subsection{Cluster expansion}
%------------------------------------------------
% \begin{frame}{General idea}
%     \begin{itemize}
%         \item represent as MPO/PEPO
%         \item cluster by size, not in $\beta$
%     \end{itemize}
% \end{frame}



\begin{frame}{General idea}
    \begin{equation}
        \mpob{1}{ {0,0}  }{}{}{}{{"",}} = \exp \left( -\beta H(\mpob{1}{}{}{}{}{{"",}})   \right)
    \end{equation}

    \begin{equation}
        \begin{split}
            \mpob{2}{ {0,1,0}  }{}{}{}{{"",}}  = \exp -\beta H( & \mpob{2}{ {,,} }{}{}{}{{"",}}) \\
            - &\mpob{2}{ {0,0,0}  }{}{}{}{{"",}}
        \end{split}
    \end{equation}

\end{frame}

\begin{frame}{General idea}

    \begin{equation}
        \begin{split}
            \mpob{3}{ {0,1,1,0}  }{}{}{}{{,,,}}  = \exp -\beta H( &\mpob{3}{ {,,,} }{}{}{}{{,,}})  \\
            -&\mpob{3}{ {0,0,0,0}  }{}{}{}{{,,,}}\\
            -&\mpob{3}{ {0,1,0,0}  }{}{}{}{{,,,}}\\
            -&\mpob{3}{ {0,0,1,0}  }{}{}{}{{,,,}}
        \end{split}
    \end{equation}

\end{frame}


%=  \exp \left(  -\beta \mpob{1}{{,,}}{}{}{}{{"H","H",}} \right)
%=  \exp \left(  -\beta \mpob{2}{{,,}}{}{}{}{{"H","H",}} \right)



\begin{frame}{Advantages}
    \begin{itemize}
        \item size extensive
        \item symmetry
        \item fast
    \end{itemize}
\end{frame}


%------------------------------------------------
%------------------------------------------------
\section{Construction 1D}
%------------------------------------------------
%------------------------------------------------

%------------------------------------------------
\subsection{Variant A}
%------------------------------------------------

\begin{frame}{Variant A}
    \begin{equation}
        \begin{split}
            &\mpob{1}{ {,}  }{}{}{}{{,,}} \\
            &\mpob{2}{ {,"1",}  }{}{}{}{{,,}}\\
            &\mpob{3}{ {,"1","1",}  }{}{}{}{{,,,}}\\
            &\mpob{4}{ {,"1","2","1",}  }{}{}{}{{,,,,,}}\\
            &\mpob{5}{ {,"1","2","2","1",}  }{}{}{}{{,,,,,}}\\
        \end{split}
    \end{equation}

\end{frame}

% %------------------------------------------------
% \subsection{Variant B}
% %------------------------------------------------
% \begin{frame}{Variant B}


%     \begin{equation}
%         \begin{split}
%             &\mpob{1}{ {,}  }{}{}{}{{,,}} \\
%             &\mpob{2}{ {,"1",}  }{}{}{}{{,,}}\\
%             &\mpob{3}{ {,"1","2",}  }{}{}{}{{,,,}}\\
%             &\mpob{4}{ {,"1","2","3",}  }{}{}{}{{,,,,,}}\\
%             &\mpob{5}{ {,"1","2","3","4",}  }{}{}{}{{,,,,,}}\\
%         \end{split}
%     \end{equation}


% \end{frame}

%------------------------------------------------
\subsection{Variant C}
%------------------------------------------------
\begin{frame}{Variant C}

    \begin{equation}
        \begin{split}
            &\mpob{1}{ {,}  }{}{}{}{{,,}} \\
            &\mpob{2}{ {,"1",}  }{}{}{}{{,,}}\\
            &\mpob{3}{ {,"1","1'",}  }{}{}{}{{,,,}}\\
            &\mpob{4}{ {,"1","2","1'",}  }{}{}{}{{,,,,,}}\\
            &\mpob{5}{ {,"1","2","2'","1'",}  }{}{}{}{{,,,,,}}\\
        \end{split}
    \end{equation}

\end{frame}
%-----------

% %------------------------------------------------
% \subsection{Comparison}
% %------------------------------------------------
% \begin{frame}
%     \begin{itemize}
%         \item bond dimension
%         \item "unwanted" chains
%     \end{itemize}
% \end{frame}

%------------------------------------------------
\subsection{Results}
%------------------------------------------------
\begin{frame}{Error measure}


    \begin{equation}
        \epsilon(\text{map}) = \frac{|| \exp -\beta H( \text{map} ) - \text{MPO}(\text{map}) || }{|| \exp -\beta H(\text{map}) || }
    \end{equation}

\end{frame}


\begin{frame}
    \begin{figure}
        \includegraphics[scale=0.6]{Figures/1D_ising.pdf}
    \end{figure}
\end{frame}


\begin{frame}
    \begin{figure}
        \includegraphics[scale=0.6]{Figures/1D_heis.pdf}
    \end{figure}
\end{frame}

\begin{frame}
    \begin{figure}
        \includegraphics[scale=0.6]{Figures/rand_01.pdf}
    \end{figure}
\end{frame}

% \begin{frame}
%     \begin{figure}
%         \includegraphics[scale=0.6]{Figures/rand_02.pdf}
%     \end{figure}
% \end{frame}

%------------------------------------------------
%------------------------------------------------
\section{Construction 2D}
%------------------------------------------------
%------------------------------------------------

%------------------------------------------------
\subsection{ Linear blocks}
%------------------------------------------------


\begin{frame}{Construction 2D: Linear blocks}
    \begin{equation}
        \mpob{1}{ {,}  }{}{}{}{{,,}}
    \end{equation}

    \begin{equation}
        \pepob{2}{2}{{"1",,}}{{,,}}{{0,0,1,1}}  \pepob{2}{2}{{,,}}{{"1",,}}{{0,1,0,1}}
    \end{equation}



\end{frame}


\begin{frame}{Construction 2D: Linear blocks}
    \begin{equation}
        \begin{split}
            \pepob{2}{2}{{"1","1",}}{{"1","1",}}{{0,0,0,1}}  \pepob{2}{2}{{"1","1",}}{{"1","1",}}{{0,0,1,0}}\pepob{2}{2}{{"1","1",}}{{"1","1",}}{{0,1,0,0}} \pepob{2}{2}{{"1","1",}}{{"1","1",}}{{1,0,0,0}}\\
            \pepob{3}{2}{{"1","1","1","1"}}{{"1","1","1","1"}}{{0,0,0,1,1,1}} \pepob{2}{3}{{"1","1","1","1"}}{{"1","1","1","1"}}{{0,1,0,1,0,1}}
        \end{split}
    \end{equation}
\end{frame}

\begin{frame}{Construction 2D: Linear blocks}
    \begin{equation}
        \pepob{3}{2}{{"1","1","1","1"}}{{"1","1","1","1"}}{{0,0,0,1,0,1}}
    \end{equation}

    \begin{equation}
        \pepob{3}{3}{{"1","1","1","1","1","1",}}{{"1","1","1","1","1","1",}}{{1,0,1,0,0,0,1,0,1}}
    \end{equation}
\end{frame}

\begin{frame}{Construction 2D: Linear blocks}
    \begin{equation}
        \begin{split}
            \mpob{4}{ {,"1","2","1",}  }{}{}{}{{,,,,,}} \Rightarrow &\pepob{3}{2}{{"2","1","2","1",}}{{"1","1","1",,}}{{0,1,1,0,0,0}}\\
            &\pepob{3}{2}{{"2","1",,}}{{"1","1",,,}}{{0,0,0,0,1,1}}
        \end{split}
    \end{equation}
    And many more "linear" blocks
\end{frame}

%------------------------------------------------
\subsection{Loops}
%------------------------------------------------
\begin{frame}{Loops}


    \def \figone {{\pepob{2}{2}{{,,,,}}{{,,,,}}{{0,0,0,0}}}}
    \def \figtwo {{\pepob{2}{2}{{"$\alpha$","$\alpha$",}}{{"$\alpha$","$\alpha$",}}{{0,0,0,0}}}}

    \begin{equation}
        \figtwo
    \end{equation}

    \begin{equation}
        \pepob{3}{2}{{"$\beta$","$\beta'$","$\beta$","$\beta'$"}}{{"$\beta$","$\alpha$","$\beta'$",,}}{{0,0,0,0,0,0}} \pepob{2}{3}{{"$\beta$","$\alpha$","$\beta'$",,}}{{"$\beta'$","$\beta$","$\beta'$","$\beta$"}}{{0,0,0,0,0,0}}
    \end{equation}

    \begin{itemize}
        \item bond dim
        \item solver: see later
    \end{itemize}

\end{frame}

\begin{frame}{Unsolved}
    \begin{equation}
        \pepob{3}{3}{{,,,,,,,,}}{{,,,,,,,,}}{{0,0,1, 0,0,0, 1,0,0,0}} \pepob{3}{3}{{,,,,,,,,}}{{,,,,,,,,}}{{1,0,0, 0,0,0, 0,0,1}}
    \end{equation}

    Easy to solve on finite lattice, difficult in thermodynamic limit...
\end{frame}


% \exp -\beta H( & \pepob{2}{2}{{,,,,}}{{,,,,}}{{0,0,0,0}})\\
%- \pepob{2}{2}{{,,,,}}{{,,,,}}{{0,0,0,0}}

%------------------------------------------------
%------------------------------------------------
\section{Solvers}
%------------------------------------------------
%------------------------------------------------



% %------------------------------------------------
% \subsection{Numerical considerations}
% %------------------------------------------------
% \begin{frame}{Numerical considerations}
%     Normalisation: PEPS $O \rightarrow O/ \alpha$
%     \begin{equation}
%         \frac{ \exp A } { \alpha^N }  =  \exp \left(   A- N \ln{\alpha} \cdot I \right)
%     \end{equation}

%     Avoid large values in tensor
% \end{frame}

% \begin{frame}{Fast cell contraction}
%     \begin{itemize}
%         \item Bottleneck: find all possible contractions of virtual levels
%         \item Solution: Construct sparse PEPO, contract geometry
%     \end{itemize}

%     \begin{figure}
%         \includegraphics[scale=0.8]{Figures/pepo_contraction.pdf}
%     \end{figure}
% \end{frame}


%------------------------------------------------
\subsection{Linear solver}
%------------------------------------------------

% \begin{frame}{Linear Solver}
%     \begin{figure}
%         \includegraphics[scale=0.6]{Figures/linprob.pdf}
%     \end{figure}
% \end{frame}

\begin{frame}{Linear solver}
    \begin{itemize}
        \item pseudoinverse
        \item optimisation for tree graphs
        \item implemented for any shape
    \end{itemize}

    \begin{equation}
        \pepob{3}{3}{{"1","1","1","1","1","1",}}{{"1","1","1","1","1","1",}}{{1,0,1,0,0,0,1,0,1}}   \mpob{4}{ {,"1","2","1",}  }{}{}{}{{,,,,,}}
    \end{equation}

    \begin{figure}
        \includegraphics[scale=0.5]{Figures/mexample.png}
    \end{figure}

\end{frame}

\begin{frame}
    \begin{figure}
        \includegraphics[scale=0.6]{Figures/ising_no_inverse_finite.pdf}
    \end{figure}
\end{frame}


%------------------------------------------------
\subsection{Non-linear solvers}
%------------------------------------------------
\begin{frame}{sequential linear}
    \begin{itemize}
        \item initialize randomly
        \item use linear sovler for 1 tensor
        \item fast
    \end{itemize}
\end{frame}


\begin{frame}{true non-linear solver}
    \begin{itemize}
        \item Matlab fsolve
        \item exact jocobian
        \item multiple patterns
        \item multiple maps
    \end{itemize}
\end{frame}

%------------------------------------------------
%------------------------------------------------
\section{2D Transversal Ising Model}
%------------------------------------------------
%------------------------------------------------


% %------------------------------------------------
% \subsection{Overview}
% %------------------------------------------------
% \begin{frame}{Overview}
%     \begin{equation}
%         \hat{H} = -J \left (  \sum_{<i j>} \sigma^z_i \sigma^z_j + \Gamma \sum_i \sigma^x_i \right )
%     \end{equation}
% \end{frame}

% \begin{frame}{Overview}
%     \begin{figure}
%         \includegraphics[scale=0.5]{Figures/2disingphase.png}
%         \caption{figure taken from \cite{PhysRevB.93.155157}}
%     \end{figure}
% \end{frame}


%------------------------------------------------
\subsection{First results}
%------------------------------------------------
\begin{frame}{First results}
    \begin{figure}
        \includegraphics[scale=0.4]{Figures/g00.pdf}
    \end{figure}
\end{frame}

% \begin{frame}{First results}
%     \begin{figure}
%         \includegraphics[scale=0.5]{Figures/g15.pdf}
%     \end{figure}
% \end{frame}

\begin{frame}{First results}
    \begin{figure}
        \includegraphics[scale=0.5]{Figures/g25.pdf}
    \end{figure}
\end{frame}

% \begin{frame}{First results}
%     \begin{figure}
%         \includegraphics[scale=0.5]{Figures/g28.pdf}
%     \end{figure}
% \end{frame}



%------------------------------------------------
%------------------------------------------------
\section{Conclusion and outlook}
%------------------------------------------------
%------------------------------------------------
\begin{frame}{Conclusion}
    \begin{itemize}
        \item Working code for 1D and 2D
        \item General solvers
        \item Promising first results in 2D
    \end{itemize}
\end{frame}


\begin{frame}{Outlook: short term}
    \begin{itemize}
        \item Accurate estimate transversal Ising quantum critical point
        \item Improve blocks for loops
        \item continuous improvements framework
    \end{itemize}
\end{frame}


\begin{frame}{Outlook: long term}
    \begin{itemize}
        \item Incorporate symmetries of Hamiltonians
        \item Look at other (types of) Hamiltonians
        \item Generalize for other lattice geometries
        \item Generalize to 3D
    \end{itemize}
\end{frame}

\begin{frame}[allowframebreaks]
    \frametitle{References}
    \bibliographystyle{elsarticle-num}
    \bibliography{bib}
\end{frame}


\end{document}

While it is often possible to find exact \Gls{MPO} representation to represent a wide class of Hamiltonians (see \cref{mpo_hamil}), it is much harder to do the same for exponentiated operators. These operators play an important role: they act as time evolution operators for quantum systems $ \ket{\Psi(t)} = \exp( -i \hat{H} t ) \ket{\Psi(0)}$. A very similar operator governs the partition function in statistical mechanics: the probability of finding a system at inverse temperature $\beta = \frac{1}{T}$ in a microstate $i$ is given by $p_i = \exp(  - \beta \hat{H_i} )$. This is often called “imaginary” time, due to the substitution $\beta = i t$. The ability to calculate these operators is essential for understanding the dynamics of a given quantum model, and making contact with real world observations of these systems at finite temperature.

\subsection{Statistical mechanics}\label{subsec:statmech}

The physics of a system in thermodynamic equilibrium can be derived from its partition function Z. The classical formula generalises to a density matrix $\rho$ as follows:
\begin{equation}
    \begin{split}
        Z &= \sum e^{ - \beta E_n} \\
        &= \sum_n \Braket{n | e^{ - \beta \hat{H} }  | n} \\
        &= \Tr( e^{ - \beta \hat{H} } )
    \end{split}
\end{equation}
The first line is the partition function for classical discrete systems. The index n runs over all possible microstates. It is known that the probability to find the system in a given microstate is given by
\begin{equation}
    p_i = \frac{\sum e^{ - \beta E_i}}{Z} .
\end{equation}
An useful quantity is the density matrix $\rho$:
\begin{equation}
    \begin{split}
        \rho &= \sum_j p_i  \Ket{ \Psi_j} \Bra{\Psi_j}   \\
        &= \sum_j \frac{ e^{ - \beta \hat{H} } }{Z}  \Ket{ \Psi_j} \Bra{\Psi_j} .
    \end{split}
\end{equation}
With this notation, the partition function Z and ensemble average of an operator $\hat{X}$ are given by:
\begin{equation}
    \begin{split}
        Z &= \Tr( \rho) \\
        \Braket{X} &= \Tr(\rho \hat{X}) .
    \end{split}
\end{equation}

\subsection{Applications}

\subsubsection{Temporal correlation functions}
The dynamical behaviour of a system can be captured by its dynamic correlation function:
\begin{equation}
    \begin{split}
        C(r,t) &= \left<  \hat{X}(0,0) \hat{X}(r,t)  \right >\\
        &=  \left<  \hat{X}(0)  e^{i \hat{H} t}  \hat{X}(r)  e^{-i \hat{H} t}   \right >
    \end{split}
\end{equation}
This requires time evolution operators $e^{-i \hat{H} t}$.

\subsubsection{Ground state}
One practical way of finding the ground state $ \ket{E_0}$ is to cool down a given random state $ \ket{\Psi(0)}$ to very small T (large $\beta$) \cite{Orus2014}:
\begin{equation}
    \ket{E_0} = \lim_{\beta \rightarrow \infty} \frac{e^{-\beta \hat{H} } \ket{\Psi(0)}  }{  \Braket{ \Psi(\beta) | \Psi(\beta) }  } ,\quad  \ket{\Psi(\beta)} =  e^{-\beta \hat{H} } \ket{\Psi(0)} .
\end{equation}

\subsection{\Gls{TN} methods}\label{rt_tn_methods}
In the following section I will give a very short review of the current \Gls{TN} methods to simulate real or imaginary time evolution. This overview is mainly based on the review paper \cite{Paeckel2019}.

% https://arxiv.org/pdf/1901.05824.pdf

\subsubsection{Approximations to \texorpdfstring{$ \hat{U}(\delta)$}{U}}
The goal is to approximately make a \Gls{MPO} for a small timestep $\delta$ which gives a new \Gls{MPS} at time $t+\delta$.
\paragraph{\Gls{TEBD}}
Time-evolving block decimation (\Gls{TEBD})  uses the Trotter-Suzuki decomposition. Suppose the chain is split in even and odd sites.
\begin{equation}
    \hat{H} = \hat{H}_{\text{even}}+\hat{H}_{\text{even}}
\end{equation}

\begin{equation}\label{trotter_exp}
    \begin{split}
        \hat{U} = e^{-i \delta \hat{H}_{\text{even}}}  e^{-i \delta \hat{H}_{\text{even}} }e^{-i \delta \left[ \hat{H}_{\text{even}}, \hat{H}_{\text{odd}} \right] }\\
        \approx e^{-i \delta \hat{H}_{\text{even}}}  e^{-i \delta \hat{H}_{\text{even}} }
    \end{split}
\end{equation}
This is now easy to solve: first apply $e^{-i \delta \hat{H}_{\text{even}}}$ for every even bond and afterwards  $e^{-i \delta \hat{H}_{\text{odd}}}$. The error is $O(\delta^2)$ and the number of steps to reach temperature $\beta$ is $\beta / \delta$. The error can be made small. This can be generalised to higher order schemes.

\paragraph{ \Gls{MPO} $W^{I,II}$}
These methods directly use the \Gls{MPO} representation of a certain Hamiltonian
%https://tensornetwork.org/mps/algorithms/timeevo/
This is a more recent method (2015) to construct an \Gls{MPO} first detailed in \cite{Zaletel2015}. The idea is to generalise
\begin{equation}
    1+ \delta \sum_x H_x \rightarrow \prod_x (1+  \delta H_x)
\end{equation}
The error is formally still $O(\delta^2)$, but includes many more terms. The advantages lay in the fact that the form above has an efficient representation as an \Gls{MPO}. \Gls{MPO} $W^I$ and $W^{II}$ are capable of dealing with long-ranged interaction terms which makes it suitable to simulate 2D systems \cite{Paeckel2019}.

\subsubsection{Global Krylov method}

Krylov methods are widely used in linear algebra to calculate eigenvectors. An example is the Lanczos algorithm. These methods are applied to \Glspl{MPS}, but do not fully make use of its structure. For this method, only a \Gls{MPO} representation is needed.

\subsubsection{MPS-local methods}

\paragraph{Local Krylov}
The Krylov methods from the previous paragraph can be adapted to work on a reduced basis.

\paragraph{TDVP} Time-dependent variational principle (TDVP) can be seen as a further development of the local Krylov method. Its also been formulated in as a tangent space algorithm, similar to the \Gls{VUMPS} derivation (\cref{vumps_Deriv}). The Schrödinger equation becomes
\begin{equation}
    i \frac{\partial \ket{ \Psi( A(t) ) }}{\partial t} = \mathcal{P}_{ A ( t) }  H  \ket{ \Psi( A (t) ) } .
\end{equation}
Where the right-hand side is projected on the tangent space, because the left-hand side is also a tangent vector. (See \cite{Vanderstraeten2019}).


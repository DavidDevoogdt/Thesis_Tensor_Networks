This chapter tested the proposed cluster expansion in a number of different ways. In \cref{sec:results1d}, the accuracy of the constructions in \cref{H4_mpo_cons} was measured against the exact exponentiation of the Hamiltonian on a cyclic chain. It was shown that taking the pseudoinverse as explained in \cref{subsec:linear_solver} is absolutely necessary to obtain a converging cluster expansion.

A surprising result is that the strict types, which only add the explicitly calculated blocks to the expansion but no longer chains, result in a less performant cluster expansion. Independent of the Hamiltonian, type A results in the smallest errors and the smallest bond dimension. The largest virtual level can be truncated to any desired bond dimension. Constructing an explicit rotationally invariant \Gls{MPO} does not result in a lower error.
Overall, the 1D results show that an order 7 construction (bond dimension 86) can represent the $e^{\beta \hat{H}}$ for larger imaginary time steps. For the transverse Ising model, $\beta_{exact} = 0.6$ while for Heisenberg it amounts to $\beta_{exact} = 0.05$.
The construction from 1D was generalised to 2D in \cref{H4_pepo_cons}. \cref{sec:results2d} presented the 2D results in a manner very similar to the 1D results. It was found that the 1D construction generalises well to 2D. Beside the equivalent blocks from 1D, also the loop contribution are important to keep the error low. Higher extensions improve upon the result, but they also require a large bond dimension. The results are somewhat less reliable than the 1D counterpart, because even with reduced density matrices, it is hard to contract a \Gls{TN} large enough to capture all the details of the model.
The fact that linear blocks and one single loop are able to capture the essence of the exponential opens up the possibility to generalise the method to 3D setting.
With the results from these sections, it is clear that this method holds some real potential. In \cref{subsec:2dpahsediag}, some phase transitions are calculated with the constructed tensor exponentials in combination with \Gls{VUMPS}. The $g=0$ and $g=2.5$ phase transitions match very well with the values found in literature. Calculation of the $T=0.7$ phase transition shows that also closer to the quantum critical point, critical behaviour is found.
The path is open to calculate the quantum critical point of transversal Ising model with the current methods.
\subsection{Phases of matter}

An important area of research is the study of the different phases of (quantum) matter. A phase is a state
of matter in which the macroscopic physical properties of the substance are uniform on a macroscopic length scale. These phase can be measured by thermodynamic function, i.e. by function of a few macroscopic parameters. \cite{Nishimori2011}. More precisely, for a given phase the properties vary as an analytic function of the macroscopic variables.

Interesting physics happens at the boundary between 2 or more distinc phases. The phase transitions were classified by Ehrenfest \cite{Jaeger1998}, who looked at the free energy across the phase boundary. If the free energy shows a discontinuity, it is called first order (or discontinuous) phase transition. Similarly, if the derivitive shows a discontinuity, it is called second order (or continuous). Higher order phase transitions are possible, and there are even examples of infinite order transitions, such as the BKT transition.

\subsection{symmetry breaking}

Sometimes, but not always, a phase transition is  related to spontaneous symmetry breaking. A state $\ket{\Psi}$ is said to be symmetric under a unitary transformation U if the state only changes by a phase factor: $ \hat{U} \ket{\Psi} = e^{i \phi} \ket{\Psi} $. A hamiltonian posesses a symmetry if it commutes with U: $ [H,U]=0$  \cite{Beekman2019}. A remarkable fact is that many ground states are not invariant under a symmetry U of the hamiltonian.

For phase transitions associated with a broken symmetry, one can define an order parameter. This parameter evalutes to 0 for the symmetric phase, but not for the sponteous broken phase.

In continuous or second-order phase transitions the order parameter increases continuously from zero as the critical temperature is traversed. The entropy also changes continuously. On the other hand, the correlation length and related energy scales diverge at the critical temperature. In fact, at the critical temperature of a second-order phase transition, scale invariance systems become scale-invariant, in the sense that physical properties no longer depend on the length (or energy) scale at which they are probed. Many symmetry-breaking phase transitions are second-order, with the onsets of superfluidity, (anti)ferromagnetism and many phases of liquid crystals as famous examples.\cite{Beekman2019}

\subsection{Universality }

Universality looks at the behaviour of the system near a continuous phase transition. These can be discribed well by so called power laws. For classical phase transitions (driven by temperature) near critical temperature $T_c$, observables $a_i$ depend in the following way on the reduced temperature $t=\frac{T-T_c}{T_c}$: $ a_i(t) \sim t^{\alpha_i}$. One would expect that the set of critical exponents ${\alpha_i}$ depends on the precise form of the hamiltonian of the system, but it turns out these exponents can be captured by a limited numer of universality classes. This means that the physics near criticality is completely understood once it is understood for one member of the class.

\subsection{critical exponents for spin systems}




\subsection{CFT}


\todo{central charge}



\Glspl{TN} were introduced from a practical point of view: what different kinds exist, how can you manipulate a \Gls{TN}, etc. Some useful algorithms are discussed. One of the reasons \Glspl{MPS} are the standard to simulate strongly correlated systems is its successful canonical form as discussed.  Density-matrix renormalization group (DMRG), first introuduced by White in 1992, is one of the most successful methods to study 1D lattices. This algorithm can be reformulated to fit in the \Gls{MPS}-formalism. \cref{Schollwock2011} The difficulty of contracting an 2D \Gls{TN} was touched upon, and some powerful algorithms are introduced, in particular VUMPS. The engineering approach was taken, mainly focussing on the diagrams and not on the mathematical rigour behind them. For \Gls{VUMPS}, the question is raised what the best way is to contract the lower half plane. An untested solution is proposed.
When the linear blocks are constructed without symmetry considerations, quite some blocks are required. For any one of the 4 legs, the number can range from 1 to M, with M the maximum virtual bond dimension. This results in $(M+1)^4$ blocks. With the solver presented in next chapter, this certainly can be done.Another possibility is to restrict \cref{2dblocksorder2} further by imposing rotation symmetry of the PEPO legs:
\begin{equation}
    \vcenter{ \hbox{\pepob{2}{2}{{"1","1"}}{{"1","1"}}{{
                        0,4,
                        1,1,
                    }} } } =     \vcenter{ \hbox{\pepob{2}{2}{{"1","1"}}{{"1","1"}}{{
                        4,0,
                        1,1,
                    }} } } = \vcenter{ \hbox{\pepob{2}{2}{{"1","1"}}{{"1","1"}}{{
                        4,1,
                        0,1,
                    }} } }= \vcenter{ \hbox{\pepob{2}{2}{{"1","1"}}{{"1","1"}}{{
                        0,1,
                        4,1,
                    }} } }
\end{equation}
In this way, only the blocks unique up to a permutation of the legs need to be solved.


% The aim of this chapter is to give a basic introduction to tensor networks from a computational viewpoint. First the graphical notation for tensors, which is ubiquitous in the field, will be introduced and explained. A incomplete classification of the different kinds of tensor networks will be discussed. Some routine tensor network manipulations are explained. Also, a selection of different algorithms, which are from a computational point of view the reason behind the success of tensor networks, are presented very briefly.

Tensor networks form a vast subject. This chapter gives a very brief introduction into the field of tensor networks. First, the graphical notation used to represent these tensors is explained. The most relevant kinds of tensor networks for this dissertation are introduced.

A practical introduction is given how to manipulate these networks. Next, a selected number of MPS algorithms is given. The focus will be to understand the intuition behind them, but not to  provide a mathematical rigorous treatment.

The next section treats the contraction of infinite dimensional 2D networks. This is chosen both to improve the understanding of tensor networks through example, and because one of the algorithms, VUMPS, will be widely used in to calculate 2D phase transition later on.
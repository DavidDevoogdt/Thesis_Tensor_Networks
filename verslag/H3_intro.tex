This chapter talks about strongly correlated matter. First, the focus will be on the different phases of matter, and phase transitions between them. There are a lot of interesting aspects related to phase transitions, not in the least universality. This principle says that the physics of many models at criticality can be captured by a limited numbers of classes, each charactered by a set of critical exponents. From a simulation point of view, the finite-size scaling method is introduced to capture these properties while using a relative small grid.
In the second section, some models are introduced, together with some known properties of these models. These models will be used to test the cluster expansions introduced later.
In a last section, an overview of operator exponentials is given. These are used both in statistical mechanics as in time evolution of a quantum state. The current methods for approximating these exponentials within the \Gls{TN} framework are discussed. The cluster expansions from this thesis, will also be a \Gls{TN} method to simulate the operator exponentials.
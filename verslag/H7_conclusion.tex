\section{Conclusion}

In \cref{chap1} the quantum many-body problem was introduced as one of the challenges facing modern physics. The problem is not related to the theory, but to the practical difficulty of calculating and simulating the properties of strongly correlated matter. A macroscopic overview of one class of techniques, tensor networks, was given.
In \cref{chap2}, a brief introduction to tensor networks was given. The focus was mainly to provide some insight in the algorithm using the graphical tensor network notation. In particular, the VUMPS algorithm was explained. Also, a technique was suggested to close the environment from below, because the current procedure only seems to work in specific cases \cref{subsec:results:loops_and_ext}.
\Cref{chap3} explains some concepts related to phases and phase transitions. The critical exponents associated with continuous phase transitions are discussed, together with a practical way called finite size scaling to obtain them from simulations. This sections also gives an overview of 2 important models in the field: the Heisenberg model and the Ising model in different dimension. Finally, the need for operator exponentials and tensor network methods to simulate time evolution are discussed.
\Cref{chap4} is the center point of this dissertation: it explains how the cluster expansions are made. This is written down in a very compact way, making abstraction of all implementation details and results.
\Cref{chap5} details the working of the 3 different solvers: a linear solver, a non-linear solver and a sequential linear solver. The linear solver must be implemented with great care in order to handle the ill conditioned inverses. This is done by calculating the pseudo-inverse instead of a full inverse. This can be done at the cost of an SVD decomposition per leg, while maintain the accuracy of inverting all the legs at once. Also, an algorithm for determining all possible virtual indices of a given map using PEPS contraction is given.  A starting point for exploring the source code is given.
\Cref{chap:results} finally quantifies the quality of the cluster expansions. This is done with respect to the exact solution in 1D and 2D. As second test, the thermal phase diagram of the transversal field Ising model is calculated. The numerical temperatures of the phase transition at constant transversal field correspond to the values in literature. A clear path is set out to calculate the quantum critical point.

\section{Outlook}

The cluster expansions have a lot of potential. In the course of one master dissertation, it is not possible to look into every possibility because a lot of groundwork need to be covered. Here are some interesting prospects:

\paragraph{Real time evolution}

Some first test in 1D \cref{subsec_rt_evo} show that the method also works for real time evolution. \Cref{rt_tn_methods} introduced some methods to calculate time evolutions with tensor networks. Crucially, all these methods come with a detailed error analysis. For instance, some methods have a proven error per step of $O \left( \frac{t}{N}^2  \right)$, resulting in an error of $O \left( \frac{t^2}{N}  \right)$ at time t. This can be made arbitrarily small.

\paragraph{Quantum critical point}

As stated in \cref{subsec:tricrit}, it should be possible to calculate the quantum critical point of the transverse Ising model using the current methods. This would allow to compare this method better with other methods in literature.

\paragraph{Lattices}

The whole thesis, a square lattice was used. This is definitely not the only choice. Other lattices could potentially result in even lower errors. Due to the generality of the solvers, this should be within reach given more time.

\paragraph{Higher dimensions}

A promising conclusion from \cref{subsec:2dpahsediag} is that a combination of linear blocks and simple loops seems sufficient to construct a good tensor network. This opens up the path to create a 3D version of the operator $e^{\beta \hat{H}}$.

\paragraph{Symmetries}

The limitations of simulation could be pushed beyond what is possible in the current framework by incorporating all symmetries available (both spatial and internal). This already exists for diagonalisation (e.g. see \cite{Wietek2018}). This could be embedded in the framework.

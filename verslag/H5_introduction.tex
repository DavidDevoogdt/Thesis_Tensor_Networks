The construction was explained in the previous chapter, but there is a lot of work that needs to be done to go from an idea to a working implementation in Matlab. The framework to do these calculations was programmed starting from scratch during the course of this thesis. This chapter aims to show some work involved.
First, the solvers, used to numerically extract the new blocks introduced in for each equation in the previous chapter, are explained. There are 3 different solvers: a linear solver based on matrix inversion, a nonlinear solver based on MATLAB fsolve and a sequential linear solver, which iteratively solves each occurring tensor with the linear solvers takes a step in that direction. The solver need to be as fast as possible, numerically stable and accurate.
The optimisation section gives more details about the framework in general. A section is dedicated to the code for generating the phase diagrams reported in the results chapter.  The source code is freely available, and \cref{sec:H5:source_code} shows very briefly how the code is structured. Finally, some limitations and possibilities relating to the framework are discussed.
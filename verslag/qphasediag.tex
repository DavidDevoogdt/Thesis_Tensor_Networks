\subsection{Ising model}

\subsubsection{Classical Ising}

The classical ising model is given by the following hamiltonian:
\begin{equation}
    H = -J \left (  \sum_{<i j>} \sigma_i \sigma_j + g\sum_i \sigma_i \right )
\end{equation}
where $<i j>$ runs over all neighbouring lattice sites. Classical refers to the fact that all the operators in the hamiltonian commute with each other. The values of $\sigma$ depends on the spin dimension. For spin $1/2$ lattices $\sigma \in {-1,+1}$.

\paragraph{1D Phase Diagram}

\paragraph{2D Phase Diagram}

\subsubsection{Quantum Ising}
In the quantum Ising model, the operators no longer commute with each other. An example is the transversal ising model given by:

\begin{equation}
    \hat{H} = -J \left (  \sum_{<i j>} \sigma^x_i \sigma^x_j + g \sum_i \sigma^z_i \right )
\end{equation}

\paragraph{1D Phase Diagram}

\paragraph{2D Phase Diagram}


\subsection{Heisenberg}

The heisenberg model is given by:

\begin{equation}
    \hat{H} =  -\left( \sum_{<i j>} J_x \sigma^z_i \sigma^z_j + J_y \sigma^y_i \sigma^y_j+ J_z \sigma^z_i \sigma^z_j + h \sum_i \sigma^z_i \right )
\end{equation}

These models have different names depending on the values of $J_{\alpha} $ with $\alpha=x,y,z$. $J_x = J_y \neq J_z = \Delta$ is called the XXZ model.

\subsection{Random}
It's also possible to construct random hamiltonians. \todo{in basis: hermitian H}

While it is often possible to find exact MPO representation to represent a wide class of hamiltonians \todo{find citation and examples}, it is much harder to do the same for exponentiated operators. These operators play an important role: they act as time evolution operators for quantum systems $ \ket{\Psi(t)} = \exp( -i \hat{H} t ) \ket{\Psi(0)}$. A very similar operator governs the partition function in statistical mechanics: the probability of finding a system at inverse temperature $\beta = \frac{1}{T}$ in a microstate i is given by $p_i\exp{ - \beta \hat{H_i} }$. This is often called "imaginary" time, due to the substitution $\beta = i t$. \todo{explain link CFT}. The ability to calculate these operators is essential for understanding the dynamics of a given quantum model, and making contact with real world obsevations of these systems at finite temperature.

\subsection{Statistical mechanics}

The physics of a system in thermodynamical equilibrium can be derived from it's partition function Z. The classical formula generalises to a density matrix $\rho$ as follows:
\begin{equation}
  \begin{split}
    Z &= \sum e^{ - \beta E_n} \\
    &= \sum_n \Braket{n | e^{ - \beta \hat{H} }  | n} \\
    &= \Tr( e^{ - \beta \hat{H} } )
  \end{split}
\end{equation}
The first line is the partition function for clasical discrete systems. The index n runs of all possible microstates. It is known that the propability to find the system in a given microstates is given by:
\begin{equation}
  p_i = \frac{\sum e^{ - \beta E_i}}{Z}
\end{equation}
An useful quantity is the density matrix $\rho$.
\begin{equation}
  \begin{split}
    \rho &= \sum_j p_i  \Ket{ \Psi_j} \Bra{\Psi_j}   \\
    &= \sum_j \frac{ e^{ - \beta \hat{H} } }{Z}  \Ket{ \Psi_j} \Bra{\Psi_j}
  \end{split}
\end{equation}
Whith this notation, the partition function Z and ensemble average of an operator $\hat{X}$ are given by:
\begin{equation}
  \begin{split}
    Z &= \Tr( \rho) \\
    \Braket{X} &= \Tr(\rho \hat{X})
  \end{split}
\end{equation}

\subsection{Time evolutions}
In quantum field theory, \todo{LSZ theorema herbekijken}, calculation of n-point correlation functions is extremely important to understand a given field theory.

\subsubsection{ground state}
One practical way of finding the ground state is cooling an intitial stata down very small T.

\subsection{Tensor network methods}
In the following section I will give a very short review of the current tensor network methods to simulate real or imaginary time evolution. This overview is mainly based on the review paper \cite{Paeckel2019}.

\cite{Paeckel2019}

% https://arxiv.org/pdf/1901.05824.pdf

\subsubsection{Approximations to  \texorpdfstring{$ \hat{U}(\delta)$}{U}   }

TEBD, MPO $W^{I,II}$

\subsubsection{global Krylov method }

\subsubsection{MPS-local methods }
local Krylov
TDVP

